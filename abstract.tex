\begin{center}
    \Large
    \textbf{Algebraic and Topological Persistence}
        
    \vspace{0.4cm}
    \large
    \textbf{Luciano Melodia}
    
    \vspace{0.9cm}
    \textbf{Abstract}
\end{center}

This work will begin with a brief introduction, inspired by an applied context that has provided the foundation for this thesis.  A substantial proportion of research in persistence theory has been conducted with the objective of applying persistent homology for the analysis of empirically-derived data. Nevertheless, we will adhere strictly to the theory of topological spaces and restrict ourselves to the provision of illustrative contexts for applications. This treatise addresses the theory of topological spaces (\S \ref{TopologicalSpaces}) and the foundations of persistence theory (\S \ref{PersistentHomology}). The subsequent discussion will address chain complexes (\S \ref{ChainComplex}) and the associated simplicial homology groups (\S \ref{SimplicialHomology}), as well as their relationship with singular homology theory (\S \ref{SingularHomology}). Moreover, we present the fundamental concepts of algebraic topology, including exact and short exact sequences (\S \ref{ExactandShortExactSequences}) and relative homology groups derived from quotienting with subspaces of a topological space (\S \ref{RelativeHomology}). These tools are used to prove the Excision Theorem (\S \ref{ExcisionTheorem}) in algebraic topology. Subsequently, the theorem is applied to demonstrate the equivalence of simplicial and singular homology for triangulable topological spaces, i.e. those topological spaces which admit a simplicial structure (\S \ref{HomologicalEquivalence}). This enables a more general theory of homology to be adopted in the study of filtrations of point clouds.

The chapter on homological persistence (\S \ref{PersistentHomology}) makes use of these tools throughout. We develop the theory of persistent homology (\S \ref{PersistentHomologyonComplexes}), the homology of filtrations of topological spaces (\S \ref{FiltrationsofComplexes}), and the corresponding dual concept of persistent cohomology (\S \ref{TheFourStandardPersistenceModules}, \S \ref{Persistent(Co)homology}). This work aims to provide mathematicians with a robust foundation for productive engagement with the aforementioned theories. The majority of the proofs have been rewritten to clarify the relationships between the techniques discussed. The novel aspect of this contribution is the canonical presentation of persistence theory and the associated ideas through a rigorous mathematical treatment for triangulable topological spaces and closing some gaps in the existing literature.