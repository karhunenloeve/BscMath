\section{Persistence Modules}
\begin{definition}{(Persistence complex)}
\label{persistencecomplex}
A \textbf{persistence complex} is an indexed family of chain complexes $\{C_\bullet^i, \partial)\}_{i \in I}$ along with chain maps $f^i: C_\bullet^i \rightarrow C_\bullet^{i+1}$.
\end{definition}

Every finite filtration $\emptyset \subset K^0 \subset K^1 \subset \cdots \subset K^m$ generates a finite persistence complex. From each of the simplicial complexes $K^i$ one can generate a chain complex $C_\bullet^i$ according to \ref{chaincomplex}. The chain map $f^i C_\bullet^i \rightarrow C_\bullet^{i+1}$ is an inclusion, since every $n$-simplex in $K^i$ is also contained in $K^{i+1}$ and the inclusion is naturally transferred to $n$-chains, since these are merely linear combinations of $n$-simplices.

\begin{figure}
\centering
\begin{tikzcd}
                                  & \vdots \arrow[d, "\partial_{n+2}"]                                   & \vdots \arrow[d, "\partial_{n+2}"]                               & \vdots \arrow[d, "\partial_{n+2}"]                                            & \vdots \arrow[d, "\partial_{n+2}"]                                   &        \\
\cdots \arrow[r, "f^{i-2}_{n+1}"] & C_{n+1}^{i-1} \arrow[r, "f^{i-1}_{n+1}"] \arrow[d, "\partial_{n+1}"] & C_{n+1}^{i} \arrow[r, "f^{i}_{n+1}"] \arrow[d, "\partial_{n+1}"] & C_{n+1}^{i+1} \arrow[r, "f^{i+1}_{n+1}", no head] \arrow[d, "\partial_{n+1}"] & C_{n+1}^{i+2} \arrow[d, "\partial_{n+1}"] \arrow[r, "f^{i+2}_{n+1}"] & \cdots \\
\cdots \arrow[r, "f^{i-2}_{n}"]   & C_{n}^{i-1} \arrow[d, "\partial_{n}"] \arrow[r, "f^{i-1}_{n}"]       & C_{n}^{i} \arrow[d, "\partial_{n}"] \arrow[r, "f^{i}_{n}"]       & C_{n}^{i+1} \arrow[d, "\partial_{n}"] \arrow[r, "f^{i+1}_{n}"]                & C_{n}^{i+2} \arrow[d, "\partial_{n}"] \arrow[r, "f^{i+2}_{n}"]       & \cdots \\
\cdots \arrow[r, "f^{i-2}_{n-1}"] & C_{n-1}^{i-1} \arrow[d, "\partial_{n-1}"] \arrow[r, "f^{i-1}_{n-1}"] & C_{n-1}^{i} \arrow[d, "\partial_{n-1}"] \arrow[r, "f^{i}_{n-1}"] & C_{n-1}^{i+1} \arrow[d, "\partial_{n-1}"] \arrow[r, "f^{i+1}_{n-1}"]          & C_{n-1}^{i+2} \arrow[d, "\partial_{n-1}"] \arrow[r, "f^{i+2}_{n-1}"] & \cdots \\
                                  & \vdots                                                               & \vdots                                                           & \vdots                                                                        & \vdots                                                               &       
\end{tikzcd}
\caption[A persistence complex]{A persistence complex, where moving to the right increases the filtration index, while moving downwards decreases the dimension.}
\end{figure}

\begin{definition}{(Persistence module \cite[§1.1]{chazal2016structure})}
A \textbf{persistence $\mathbb{R}$-module $\mathbb{V}$} is an indexed family of vector spaces $(V_r \ \vert \ r \in \mathbb{R})$ and a doubly-indexed family of linear maps $(v_s^t: V_s \rightarrow V_t \ \vert \ s \leq t)$ which satisfy the composition law $v_s^t \circ v_r^s = v_r^t$ whenever $r \leq s \leq t$, and where $v_t^t = \text{id}_{V_t}$. 
\end{definition}

\begin{remark}
In general, we can define a persistence module over any partially ordered set $T$ instead of the real numbers. This is called a {\bfseries $T$-persistence module}.
\end{remark}

\begin{example}
Consider the following persistence modules:
\begin{enumerate}
	\item Let $X$ be a topological space and let $f: X \rightarrow \R$ be a function. Consider the sublevelsets $X^t := (X,f)^t = \{ x \in X \ \vert \ f(x) \leq t \}$. The inclusion maps $(i_s^t: X^s \rightarrow X^t \ \vert \ s \leq t)$ satisfy trivially the composition law $i_s^t \circ i_r^s = i_r^t$ whenever $r \leq s \leq t$ with $i_t^t$ the identity on $X^t$. This information is also called sublevelset filtration of $(X,f)$ and is denoted as $\X_{\text{sub}}^f$ \cite[p.7]{chazal2016structure}.
	\item In general one obtains a persistence module by applying any functor from topological spaces to vector spaces. For example, let $H := H_p(-;\F)$ be the functor of the $p$-dimensional singular homology on the category of topological spaces with coefficients in some abelian group $G$, in our case in some field $\F$. We define a persistence module $\V$ by setting $V_t = H(X^t)$, and $v_s^t = H(i_s^t): H(X^s) \rightarrow H(X^t)$, which are the maps on homology induced by the inclusion maps \cite[p.7]{chazal2016structure}.

	In brief we yield $\V = H(\X_{\text{sub}}^f)$.
\end{enumerate}
\end{example}

\section{Interval Decompositions}
\label{intervaldecomposition}

\section{Persistence Barcodes}
\label{persistencebarcodes}

\section{Persistence Diagrams}
\label{persistencediagrams}

\section{Interval-indecomposable Persistence Modules}
\label{intervaldinecomposition}

\section{Zigzag Persistence}
Instead of a persistence module, zigzag persistences are always considered when we look at a zigzag diagram of a topological space or vector space, i.e. a sequence of spaces $V_1, \ldots, V_n$ where each adjacency pair is connected by a mapping $V_i \rightarrow V_{i+1}$ or $V_i \leftarrow V_{i+1}$. This means that the persistence modules discussed so far are also zigzag persistence modules in which the morphisms all point in one direction.

\begin{definition}{(Zigzag Module) \cite[§2.1]{}}

\end{definition}
