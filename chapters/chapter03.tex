\section{Persistent Homology}
\subsection{Filtrations of Complexes}
Individual (co)homology groups are consistently defined with coefficients in a
fixed field $\mathbb{F}$, organizing persistent (co)homology as a graded module
over $\mathbb{F}[x]$, as can be seen in the next chapter \ref{}. Replacing
$\mathbb{F}$ with the ring $\mathbb{Z}$ leads to substantial issues, but most of
the results still hold for principal ideal domains, as discussed in
\cite[\S 3.1]{zomorodian2004computing}.

We investigate the persistent topology of filtered topological spaces, with a primary
focus on the prototypical example of a filtered cell complex. This structure is
defined by a sequence $\mathcal{X}$ of cell complexes:
\begin{equation*}
	\mathcal{X}: \emptyset \subset X_{1} \subset X_{2} \subset \cdots \subset X_{n}
	= X_{\infty},
\end{equation*}
where $X_{1}$ starts with a single vertex $\sigma_{1}$, and each subsequent
complex $X_{i}$ is constructed by adding a single cell to the previous complex:
\begin{equation*}
	X_{i} := X_{i-1}\cup \sigma_{i}.
\end{equation*}
Here, the indexing set is $\{1, 2, \ldots, n\}$. Additionally, associated real
values $a_{i}$ are assigned to these indices, satisfying $a_{1} \leq a_{2} \leq \cdots
\leq a_{n}.$ This formulation clearly delineates the stepwise enlargement of the
complex, illustrating the dynamic evolution of its topology as new cells are
incrementally incorporated. We will use the same running example as in \cite[\S 2.2]{de2011dualities}.

\begin{example}
	\label{filteredsphere} The chosen illustrative example involves a cellular filtration\index{filtration}
	of the 2-sphere, denoted by $\mathcal{S}^{2}$. This process constructs a cell complex
	and introduces an ordering among the cells to facilitate differentiation between
	them. The development of the filtration can be articulated as follows
	\cite[\S 2.2]{de2011dualities}:

	\begin{itemize}
		\item[$\mathcal{S}^{2}:$] $\emptyset$

		\item[$\subset$] $S_{1} = \{\sigma_{1}\}$

		\item[$\subset$] $S_{2} = \{\sigma_{1}, \sigma_{2}\}$

		\item[$\subset$] $S_{3} = \{\sigma_{1}, \sigma_{2}, \sigma_{3} := (\sigma_{1}
			, \sigma_{2})\}$

		\item[$\subset$] $S_{4} = \{\sigma_{1}, \sigma_{2}, \sigma_{3}, \sigma_{4} :=
			(\sigma_{2}, \sigma_{1})\}$

		\item[$\subset$] $S_{5} = \{\sigma_{1}, \sigma_{2}, \sigma_{3}, \sigma_{4}, \sigma
			_{5} := (\sigma_{3}, \sigma_{4})\}$

		\item[$\subset$] $S_{6} = \{\sigma_{1}, \sigma_{2}, \sigma_{3}, \sigma_{4}, \sigma
			_{5}, \sigma_{6} := (\sigma_{4}, \sigma_{3})\}.$
	\end{itemize}

	In this sequence, the cell $\sigma_{1}$ symbolizes the initial point. The set $S
	_{2}$ incorporates two distinct points. In $S_{3}$, a path connecting $\sigma_{1}$
	and $\sigma_{2}$ is introduced. $S_{4}$ augments this path with its
	reverse—from $\sigma_{2}$ to $\sigma_{1}$—clearly distinguishing it from $\sigma
	_{3}$. Finally, $S_{5}$ and $S_{6}$ progressively differentiate between cells that
	represent the upper and lower halves of the sphere, respectively.
\end{example}

\subsection{Persistent Homology on Complexes}
In the context of algebraic topology, applying the homology functor $\mathsf{H}$
to a filtration of a complex $\mathcal{X}$ yields a sequence of algebraic structures:
\begin{equation}
	\mathsf{H}(\mathcal{X}): \quad \mathsf{H}(X_{1}) \to \mathsf{H}(X_{2}) \to \cdots
	\to \mathsf{H}(X_{n}),
\end{equation}
where $\mathsf{H}(-)$ generally represents either the $k$-th dimensional
homology, denoted by $H_{k}(-;\mathbb{F})$, or the total homology, expressed as $H
_{\bullet}(-;\mathbb{F})$. Here, this diagram characterizes a sequence of abelian
groups or finite-dimensional vector spaces, interconnected through vector space
homomorphisms, forming what is known as a persistence module.

Persistence modules are central to understanding how the features of a space evolve
over time. They can typically be decomposed into a direct sum of interval
modules (\ref{intervaldecomposition}). Each interval module is associated with an
ordered pair of integers $(p,q)$ where $1 \leq p \leq q \leq n$, within a finite
filtration. These pairs $(p,q)$ signify topological features that persist over an
index set $I := \{p, \ldots, q\}$, where $\inf\{I\} = p$ and $\sup\{I\} = q$. Conventionally,
these tuples are interpreted as half-open intervals $[a_{p}, a_{q+1})$, with $a_{n+1}
= \infty$ being a customary notation when the sequence extends beyond the largest
indexed space.

The decomposition of a persistence module into its constituent interval modules is
represented in a persistence diagram (\ref{persistencediagrams}), or a barcode (\ref{persistencebarcodes}).
This barcode is a multiset of ordered tuples $(p,q)$ or, alternatively, a multiset
of half-open intervals $[a_{p}, a_{q+1})$. This collection is formally expressed
through the forgetful functor $\textbf{Pers}(-)$:
\begin{align}
	\textbf{Pers}(\mathsf{H}(\mathcal{X})) & = \{(p_{1},q_{1}), \ldots, (p_{m},q_{m})\}                    \\
	                                       & \cong \{[a_{p_1}, a_{q_1+1}), \ldots, [a_{p_m}, a_{q_m+1})\}.
\end{align}
In practical applications, intervals where $a_{p} = a_{q+1}$ are usually omitted,
as they represent ephemeral topological features.

\begin{example}
	In the further elaboration of the example previously cited, which is also
	described in \ref{filteredsphere}, we consider the topological subspaces $S_{1}
	, S_{3}, S_{5}$, all of which are contractible. Meanwhile,
	$S_{2}, S_{4}, S_{6}$ are homeomorphic to the $0$-sphere, $1$-sphere, and $2$-sphere,
	respectively. This structural distinction leads to four distinct intervals in
	the persistence diagram of the total homology of a sphere, specifically $\mathcal{S}
	^{2}$:
	\begin{align}
		\textbf{Pers}(H_{\bullet}(\mathcal{S}^{2})) & = \{(1,6)_{0}, (2,2)_{0}, (4,4)_{1}, (6,6)_{2} \}             \\
		                                            & = \{(1,\infty)_{0}, (2,3)_{0}, (4,5)_{1}, (6, \infty)_{2} \}.
	\end{align}
	Here, the subscript $k$ in $(p,q)_{k}$ or $[a_{p}, a_{q+1})_{k}$ denotes a topological
	feature in the $k$-dimensional homology.
\end{example}

\subsection{The Four Standard Persistence Modules}
\label{standardpersistencemodules} The standard module of persistent homology, $H
_{\bullet}(\mathcal{X})$, illustrates how the absolute homology groups
$H_{\bullet}(X_{i})$ relate to each other as the index $i$ changes. Similar observations
can be made by considering the absolute cohomology groups $H^{\bullet}(X_{i})$,
the relative homology groups $H_{\bullet}(X_{n}, X_{i})$, and the relative cohomology
groups $H^{\bullet}(X_{n}, X_{i})$ \cite[\S 2.4]{de2011dualities}.

The persistence diagram for absolute cohomology is represented as a multiset of integer
pairs $(p,q)$, where $1 \leq p \leq q \leq n$ for a finite filtration. For relative
homology and cohomology, the persistence diagram consists of a multiset of
tuples $(p,q)$ where $0 \leq p \leq q \leq n-1$ for a finite filtration. In each
case, we interpret $(p,q)$ as a half-open interval $[a_{p}, a_{q+1})$ with the
convention that $a_{0} = -\infty$ and $a_{n+1}= \infty$
\cite[\S 2.4]{de2011dualities}.

\begin{figure}
	\begin{align*}
		H_{\bullet}(\mathcal{X})             & : \quad H_{\bullet}(X_{1}) \rightarrow \cdots \rightarrow H_{\bullet}(X_{n-1}) \rightarrow H_{\bullet}(X_{n}),             \\
		H^{\bullet}(\mathcal{X})             & : \quad H^{\bullet}(X_{1}) \leftarrow \cdots \leftarrow H^{\bullet}(X_{n-1}) \leftarrow H^{\bullet}(X_{n}),                \\
		H_{\bullet}(X_{\infty}, \mathcal{X}) & : \quad H_{\bullet}(X_{n}) \rightarrow H_{\bullet}(X_{n},X_{1}) \rightarrow \cdots \rightarrow H_{\bullet}(X_{n},X_{n-1}), \\
		H^{\bullet}(H_{\infty}, \mathcal{X}) & : \quad H^{\bullet}(X_{n}) \leftarrow H^{\bullet}(X_{n},X_{1}) \leftarrow \cdots \leftarrow H^{\bullet}(X_{n}, X_{n-1}).
	\end{align*}
	\caption{The four standard persistence modules.}
\end{figure}

\begin{example}
	For $\mathcal{S}^{2}$ we yield
	\begin{align}
		\textbf{Pers}(H_{\bullet}(S_{6},\mathcal{S}^{2}) & = \{(0,0)_{0}, (2,2)_{1}, (4,4)_{2}, (0,5)_{2}\}               \\
		                                                 & = \{[-\infty, 1)_{0}, [2,3)_{1}, [4,5)_{2}, [-\infty,6)_{3}\}.
	\end{align}
	At index $2$ there is a nontrivial element of $H_{1}(S_{6},S_{2})$ represented
	by any arc connecting the two points of $S_{2}$ \cite[\S 2.4]{de2011dualities} --
	the homology class is $[\sigma_{3}] = [\sigma_{4}]$. This class vanishes in $H_{1}
	(S_{6},S_{3})$, thus we yield the interval $[2,3)$.
\end{example}

\section{Persistent (Co)homology}
Inverse problems primarily involve inferring geometric shapes from measurements like
path integrals. Classical methods such as Fourier transforms provide extensive
information but struggle with nonlinearity and ill-posed conditions, requiring substantial
regularization. Topology, particularly through persistent homology, offers
alternative methods for deducing topological rather than geometric information. This
approach is especially useful in high-dimensional, discrete sets of points, exemplified
in the finite case by geological sonar to detect subterranean features based on density
variations \cite[\S 1]{de2011dualities}. Persistent homology identifies
topological features represented as intervals in a barcode or persistence
diagram, crucial for understanding the presence and persistence of features such
as holes or voids in topological spaces. This method is statistically robust and
can provide both qualitative and quantitative insights into point sets, which we
suspect to lie on some compact topological object \cite{chazal2014persistence,chazal2009proximity}.

In our work, we continue to examine the consequences of absolute and relative
homology and cohomology groups for filtrations of cell complexes, such as the introduced
simplicial complexes. In particular, we derive the theory in the context of filtered
simplicial complexed upon sets of points, embedded into some metric space. While
we can apply the entire theory to simplicial complexes and their (co)homology, cell
complexes provide a much broader context and significantly simplify notation in many
instances.

In particular, there are at least four naturally arising persistent objects that
can be extracted from a filtration of any topological space. We follow the results
of de Silva, Morozov, and Vejdemo-Johansson for our explanations \cite[\S 1]{de2011dualities}.
They are

\begin{equation*}
	\text{persistent}
	\begin{Bmatrix}
		\text{absolute} \\
		\text{relative}
	\end{Bmatrix}
	\begin{Bmatrix}
		\text{homology}   \\
		\text{cohomology}
	\end{Bmatrix}.
\end{equation*}

In this work, we address the computation of barcodes for all four types of persistent
objects. We demonstrate that both absolute and relative (co)homologies yield identical
barcodes and that transitions between these states are facilitated by established
duality principles. The duality between homology and cohomology is akin to the
duality in vector spaces, whereas a global duality specific to persistent topology
allows for a unique interchange:
\begin{align*}
	\text{Absolute homology}   & \rightleftarrows \text{relative cohomology.} \\
	\text{Absolute cohomology} & \rightleftarrows \text{relative homology.}
\end{align*}

The main results from the literature suggest that a single calculation is sufficient
to compute all four persistent objects due to the commutative nature of the global
duality.

\subsection{Barcode Isomorphisms}
We characterise the multisets for persistence modules that are decomposable into
interval modules. The persistence diagram partitions into $\textbf{Pers}_{0}$,
comprising finite intervals $[a, b)$ as per \cite[\S 2.3]{de2011dualities}, and $\textbf
{Pers}_{\infty}$, consisting of intervals $[a, \infty)$. This leads to the decomposition
$\textbf{Pers}= \textbf{Pers}_{0} \cup \textbf{Pers}_{\infty}$.

In this chapter, we establish that persistent homology and cohomology yield the
same intervals, or barcodes, for both absolute and relative (co-)homology
frameworks. This equivalence necessitates invoking the universal coefficient theorem
from algebraic topology, which we will proof beforehand. The universal coefficient
theorem for cohomology elegantly ties together the cohomology of a space with coefficients
in any abelian group $G$ to the homology of the space with integer coefficients.
Specifically, as notation for this proof, $H_{n}(X;\mathbb{Z})$ and
$H^{n}(X;\mathbb{Z})$ denote the $n$-th singular homology and cohomology groups
with coefficients in $\mathbb{Z}$, respectively. We further involve
$\text{Hom}(A, G)$, denoting group homomorphisms from an abelian group $A$ to another
abelian group $G$, and $\text{Ext}^{1}(A, G)$ \cite{}, which measures obstructions
in the splitting of a short exact sequence of abelian groups. Further, we use the
properties of derived functors \cite{}.

\begin{theorem}
	{(Universal Coefficient Theorem for Cohomology) \cite[\S 3.1]{hatcher2005algebraic}}
	\label{universalcoefficients} Let $X$ be a topological space and $G$ be an abelian
	group. For any integer $n \geq 0$, there is a short exact sequence:
	\begin{align*}
		0 \rightarrow \text{Ext}^{1}(H_{n-1}(X;\mathbb{Z},G) \rightarrow H^{n}(X;G) \rightarrow \text{Hom}(H_{n}(X;\mathbb{Z},G) \rightarrow 0,
	\end{align*}
	which splits, though not canonically.
\end{theorem}

\begin{proof}
	Let $C_{\bullet}(X)$ be the singular chain complex of a topological space $X$ with
	integer coefficients. The homology groups $H_{d}(X; \mathbb{Z})$ are defined
	as
	$H_{d}(X; \mathbb{Z}) := \ker(\partial_{d}) / \operatorname{im}(\partial_{d+1})$,
	where $\partial_{d}$ are the boundary maps in $C_{\bullet}(X)$. The chain
	group $C_{d}(X)$ consists of formal sums of singular $d$-simplices in $X$ with
	integer coefficients. The boundary maps $\partial_{d}: C_{d}(X) \rightarrow C_{d-1}
	(X)$ are defined by
	\[
		\partial_{d}(\sigma) = \sum_{i=0}^{d}(-1)^{i} \sigma|_{[v_0, \ldots, \hat{v}_i,
		\ldots, v_d]},
	\]
	where $\sigma: \Delta^{d} \rightarrow X$ is a singular simplex, and
	$\sigma|_{[v_0, \ldots, \hat{v}_i, \ldots, v_d]}$ denotes the restriction of
	$\sigma$ to the $i$-th face of the simplex, omitting the $i$-th vertex.

	The functor $\operatorname{Hom}(-, G)$ applied to $C_{d}(X)$ yields a group $\operatorname{Hom}
	(C_{d}(X), G)$, and the coboundary $\delta^{d}$ for the cochain complex
	$\operatorname{Hom}(C_{\bullet}(X), G)$ is defined by
	\[
		\delta^{d}(f) = f \circ \partial_{d+1}
	\]
	for $f$ in $\operatorname{Hom}(C_{d+1}(X), G)$. This leads to the cohomology groups
	$H^{d}(X; G) = \ker(\delta^{d}) / \operatorname{im}(\delta^{d-1})$. We
	consider the projective resolution of $C_{\bullet}(X)$ to effectively apply
	the $\operatorname{Ext}$ functor. Recall that for any abelian group $A$,
	\[
		\operatorname{Ext}^{1}(A, G) = R^{1} \operatorname{Hom}(A, G),
	\]
	where $R^{1}$ denotes the first right derived functor of $\operatorname{Hom}$.
	To show this equality, we begin by taking a projective resolution of $A$:
	\[
		\cdots \to P_{2} \to P_{1} \to P_{0} \to A \to 0,
	\]
	where each $P_{i}$ is a projective abelian group. Then we can apply the
	functor $\operatorname{Hom}(-, G)$ to the projective resolution:
	\[
		0 \to \operatorname{Hom}(P_{0}, G) \to \operatorname{Hom}(P_{1}, G) \to \operatorname{Hom}
		(P_{2}, G) \to \cdots.
	\]
	This sequence is exact on the left because $\operatorname{Hom}(-, G)$ is left exact
	and each $P_{i}$ is projective. The first right derived functor
	$R^{1}\operatorname{Hom}(A, G)$ is defined as the cohomology of this sequence
	at the location corresponding to $P_{1}$:
	\[
		R^{1} \operatorname{Hom}(A, G) = \frac{\ker(\operatorname{Hom}(P_{1}, G) \to
		\operatorname{Hom}(P_{2}, G))}{\operatorname{im}(\operatorname{Hom}(P_{0}, G)
		\to \operatorname{Hom}(P_{1}, G))}.
	\]
	The group $\operatorname{Ext}^{1}(A, G)$ classifies extensions of $A$ by $G$,
	equivalent to the kernel/image calculation in the cohomology of the
	$\text{Hom}$-sequence, confirming
	\[
		\operatorname{Ext}^{1}(A, G) = R^{1} \operatorname{Hom}(A, G).
	\]

	Given $H_{d}(X; \mathbb{Z})$, consider the short exact sequence obtained from the
	projective resolution of $\mathbb{Z}$:
	\[
		0 \rightarrow \mathbb{Z}\rightarrow F \rightarrow H_{d}(X; \mathbb{Z}) \rightarrow
		0,
	\]
	where $F$ is free. Applying $\operatorname{Hom}(-, G)$ gives
	\[
		0 \rightarrow \operatorname{Hom}(H_{d}(X; \mathbb{Z}), G) \rightarrow \operatorname{Hom}
		(F, G) \rightarrow \operatorname{Hom}(\mathbb{Z}, G) \rightarrow \operatorname{Ext}
		^{1}(H_{d}(X; \mathbb{Z}), G) \rightarrow 0.
	\]
	We apply $\operatorname{Ext}^{\bullet}$, which gives rise to the long exact
	sequence of $\operatorname{Ext}$ groups:
	\[
		0 \rightarrow \operatorname{Hom}(H_{d}(X; \mathbb{Z}), G) \rightarrow H^{d}(X
		; G) \rightarrow \operatorname{Ext}^{1}(H_{d-1}(X; \mathbb{Z}), G) \rightarrow
		0.
	\]
	Finally, we verify exactness and splitting. The term $\operatorname{Ext}^{1}(H_{d-1}
	(X; \mathbb{Z}), G)$ measures the non-trivial extensions of $G$ by $H_{d-1}(X;
	\mathbb{Z})$, which corresponds to the obstructions to lifting
	$H_{d-1}(X; \mathbb{Z})$ linearly over $G$. The term $\operatorname{Hom}(H_{d}(
	X; \mathbb{Z}), G)$ represents the group homomorphisms from $H_{d}(X; \mathbb{Z}
	)$ to $G$, which naturally includes in $H^{d}(X; G)$. The sequence is exact at
	each stage by the properties of derived functors and their application to the singular
	chain complex. The sequence ends with $0$ because $\operatorname{Ext}^{1}$ of
	a projective (or free) module vanishes, and $\mathbb{Z}$ is free. The sequence
	splits because the functor $\operatorname{Hom}(-, G)$ preserves products and
	coproducts. However, the way it splits is not canonical and depends on the
	choice of a splitting homomorphism, which is not unique.

	In the generalization of the universal coefficient theorem to the case of modules
	over a principal ideal domain, the $\text{Ext}^{1}$ terms vanish since $\F$ is
	a field, so we obtain isomorphisms $H^{d}(X;\F) \cong \Hom(H_{d}(X;\F),\F)$ \cite[p.198
	\S 3.3.1]{hatcher2005algebraic}.
\end{proof}

\begin{theorem}
	For all integers $d \geq 0$ it holds that \cite[\S 2.3]{de2011dualities}:
	\begin{align*}
		\textbf{Pers}(H_{d}(\mathcal{X}) = \textbf{Pers}(H^{d}(\mathcal{X}),                         \\
		\textbf{Pers}(H_{d}(X_{\infty}, \mathcal{X}) = \textbf{Pers}(H^{d}(X_{\infty}, \mathcal{X}).
	\end{align*}
\end{theorem}

\begin{proof}
	When we consider coefficients in a field rather than in a ring, the universal
	coefficient theorem assures us of a natural isomorphism between the $d$-th
	cohomology group and the homomorphisms from the $d$-th homology group to the
	base field:
	\begin{equation}
		H^{d}(X;\mathbb{F}) \cong \text{Hom}(H_{d}(X;\mathbb{F}),\mathbb{F}).
	\end{equation}
	Therefore, the associated maps
	\begin{equation}
		H_{d}(X_{i};\mathbb{F}) \rightarrow H_{d}(X_{j};\mathbb{F}) \quad \text{and}\quad
		H^{d}(X_{i};\mathbb{F}) \leftarrow H^{d}(X_{j};\mathbb{F})
	\end{equation}
	are adjoint and hence possess the same rank. Since the persistence intervals over
	a field are uniquely determined by the dimension and rank of the homology vector
	spaces, it follows that this holds for both homology and cohomology. Consequently,
	they share the same barcode.
\end{proof}

Consider a filtration $X_{1} \subseteq X_{2} \subseteq \cdots \subseteq X_{n}$ of
a topological space $X$, extending to $X_{\infty}$ where $X_{\infty}$ is the
direct limit of the filtration. The homology groups $H_{d}(X_{n})$ for some fixed
dimension $d$ serve as the initial terms for the relative homology groups $H_{d}(
X_{\infty}, X)$. Since $H_{d}(X)$ is consistent with $H_{d}(X_{n})$ as $n$
approaches infinity, these sequences can be unified into a single sequence:
$H_{d}(X) \to H_{d}(X_{\infty}, X)$. For this concatenated sequence, the indices
are denoted as
$\{1, 2, \ldots, n = 0^{\flat}, 1^{\flat}, 2^{\flat}, \ldots, (n-1)^{\flat}\}$, using
the $\flat$ symbol to indicate terms in the relative homology part of the sequence.

This structure allows us to discuss the persistence diagram associated with this
homological configuration. The persistence intervals in this diagram can
generally be categorized into three types:

\begin{itemize}
	\item Intervals of the form $(p, q)$ where $1 \leq p \leq q < n$, denoted as
		$[p, q+1)$ or $[a_{p}, a_{q+1})$.

	\item Intervals of the form $(p^{\flat}, q^{\flat})$ where $0 < p \leq q \leq n
		-1$, denoted as $[p^{\flat}, q^{\flat}+1)$ or $[a_{p^\flat}, a_{q^\flat+1})$.

	\item Intervals of the form $(p, q^{\flat})$ where $1 \leq p \leq n$ and $0 \leq
		q \leq n-1$, represented as $[p, q^{\flat}+1)$ or $[a_{p}, a_{q^\flat+1})$.
\end{itemize}

\begin{corollary}
	The barcode $\textbf{Pers}(H_{d}(\X) \rightarrow H_{d}(X_{\infty}, \X)$ comprises
	the following collections of intervals \cite[\S 2.5]{de2011dualities}:
	\begin{itemize}
		\item An interval $[a,b)$ for every interval $[a,b)$ in $\textbf{Pers}_{0}(H_{d}
			(\X))$.

		\item An interval $[a^{\flat}, b^{\flat})$ for every interval $[a,b)$ in $\textbf
			{Pers}_{0}(H_{d-1}(\X))$.

		\item An interval $[a,a^{\flat})$ for every interval $[a,\infty)$ in $\textbf
			{Pers}_{\infty}(H_{d}(\X))$.
	\end{itemize}
\end{corollary}

\begin{proof}
	We begin by analyzing the first two types of intervals in the persistence diagram
	$\textbf{Pers}(H_{d}(X) \to H_{d}(X_{\infty}, X))$. These intervals either do
	not intersect the intermediate term $H_{d}(X_{n})$ or terminate before it. Consequently,
	they correspond precisely to the finite intervals in $\textbf{Pers}(H_{d}(X))$
	and $\textbf{Pers}(H_{d}(X_{\infty}, X))$, clarifying the first two cases. The
	correspondence
	$\textbf{Pers}_{0}(H_{d}(X_{\infty}, X)) = \textbf{Pers}_{0}(H_{d-1}(X))$
	helps in mapping these relationships.

	The third case requires examining intervals of the form $[a, b^{\flat})$ and
	proving that they are invariably of the form $[a, a^{\flat})$, which means that
	the paired intervals $[a,\infty)$ and $[-\infty,a)$ in $\textbf{Pers}_{\infty}(
	H_{d}(\X))$ and $\textbf{Pers}_{\infty}(H_{d}(X_{\infty}, \X))$ are restrictions
	of a single interval $[a,a^{\flat})$ in the concatenated sequence \cite[p.6]{de2011dualities}.
	To establish this, we compare the ascending filtration defined by the images of
	$H_{d}(X_{i})$ in $H_{d}(X_{n})$ for $i = 1, 2, \ldots, n-1$ (denoted as
	$\text{Im}(H_{d}(X_{i}) \to H_{d}(X_{n}))$) with the descending filtration defined
	by the kernels of $H_{d}(X_{n})$ in $H_{d}(X_{\infty}, X)$ for $i = 1, 2, \ldots
	, n-1$ (denoted as $\text{Ker}(H_{d}(X_{n}) \to H_{d}(X_{n}, X_{i}))$). This
	examination revolves around the fundamental properties of the homology groups
	in a filtration setting.

	For each index $i$, the image and kernel correspond to the same subspace of $H_{d}
	(X_{n})$. This equivalence is guaranteed by the homology long exact sequence
	associated with the pair $(X_{n}, X_{i})$, which links the relative and absolute
	homology groups. Specifically, the exact sequence implies that any cycle in $\text{Im}
	(H_{d}(X_{i}) \to H_{d}(X_{n}))$ that becomes a boundary in $H_{d}(X_{n}, X_{i}
	)$ must vanish, thus equating the image and kernel. As a result, both
	filtrations align perfectly, establishing that the third type of interval indeed
	maps to self-closing intervals of the form $[a, a^{\flat})$.
\end{proof}

\subsection{Persistent Chain Complexes}
Alternatively, following \cite[\S 2.6]{de2011dualities}, the standard persistence module
can be described via a filtered cell complex $\mathcal{X}:= \sigma_{1} \cup \cdots
\cup \sigma_{d}$, so that the persistence module can be expressed as the
sequence
\[
	\mathcal{C}: \quad C_{1} \to C_{2} \to \cdots \to C_{d},
\]
for a finite filtration, where each $C_{i} := \langle \sigma_{1}, \ldots, \sigma_{i}
\rangle$ is a vector space (or an abelian group) over a field $\mathbb{F}$,
generated by the elements $\sigma_{1}, \ldots, \sigma_{i}$. The boundary operator
is defined by $\partial \sigma_{j} = \sum_{i<j}\lambda_{ij}\sigma_{i}$ for $\lambda
_{ij}\in \mathbb{F}$ for all $i, j \in \{1, \ldots, d\}$.

Then $C_{\bullet}(X_{i}) = (C_{i},\partial_{i})$ is the chain complex for the absolute
homology of $X_{i}$, and $C_{\bullet}(\mathcal{X}) = (\mathcal{C},\partial)$
represents the persistent version for $\mathcal{X}$. This leads to the definition
of the persistent absolute homology of $\mathcal{X}$ as
\[
	H_{\bullet}(\mathcal{X}) = H(\mathcal{C}, \partial): \quad \frac{\ker(\partial_{1})}{\text{im}(\partial_{1})}
	\to \frac{\ker(\partial_{2})}{\text{im}(\partial_{2})}\to \cdots \to \frac{\ker(\partial_{d})}{\text{im}(\partial_{d})}
	.
\]

For the persistent absolute cohomology $H^{\bullet}(\mathcal{X})$, we define
\[
	\mathcal{C}^{\flat}: \quad C^{\flat}_{1} \leftarrow C^{\flat}_{2} \leftarrow \cdots
	\leftarrow C^{\flat}_{d},
\]
where $C_{i}^{\flat} = \text{Hom}(C_{i}, \mathbb{F}) = \langle \sigma_{1}^{\flat}
, \sigma_{2}^{\flat}, \ldots, \sigma_{i}^{\flat} \rangle$, with
$\{\sigma_{i}^{\flat}\}$ as the dual basis of $\{\sigma_{i}\}$. The coboundary $\delta
= \partial^{\flat}$ is defined as the adjoint of $\partial$. Then
$C^{\flat}(\mathcal{X}) = (\mathcal{C}^{\flat}, \delta)$ and
$H^{\bullet}(\mathcal{X}) = H(\mathcal{C}^{\flat}, \delta) = \ker(\delta) / \text{im}
(\delta)$. This too constitutes a persistence module, with morphisms in the
reverse direction.

\begin{example}
	For $\mathcal{S}^{2}$, the boundary operator is given by
	\begin{align}
		\partial\sigma_{1} & = \partial\sigma_{2} = 0,                       \\
		\partial\sigma_{3} & = \partial\sigma_{4} = \sigma_{1}-\sigma_{2},   \\
		\partial\sigma_{5} & = \partial\sigma_{6} = \sigma_{3} - \sigma_{4}.
	\end{align}
	This information can also be summarized in a matrix. Similarly, the coboundary
	operator is given by
	\begin{align}
		\delta\sigma_{1}^{\flat} = -\delta\sigma_{2}^{\flat} & = \sigma_{3}^{\flat} + \sigma_{4}^{\flat}, \\
		\delta\sigma_{3}^{\flat} = -\delta\sigma_{4}^{\flat} & = \sigma_{5}^{\flat} + \sigma_{6}^{\flat}, \\
		\delta\sigma_{5}^{\flat} = \delta\sigma_{6}^{\flat}  & = 0.
	\end{align}
	This data can analogously be summarized in a transposed matrix
	\cite[p.7]{de2011dualities}.

	The relative homology and cohomology persistence modules are then defined as
	the homology of the persistence modules \cite[p.7]{de2011dualities}:
	\begin{align}
		(C_{d}/\mathcal{C}):         & \quad C_{d} \rightarrow (C_{d}/C_{1}) \rightarrow (C_{d}/C_{2}) \rightarrow \cdots \rightarrow (C_{d}/C_{d-1}),                             \\
		(C_{d}/\mathcal{C})^{\flat}: & \quad C_{d}^{\flat} \leftarrow (C_{d}/C_{1})^{\flat} \leftarrow (C_{d}/C_{2})^{\flat} \leftarrow \cdots \leftarrow (C_{d}/C_{d-1})^{\flat}.
	\end{align}
	We note that the mappings $\rightarrow$ from $\mathcal{C}$ and the mappings $\leftarrow$
	from $(C_{d}/\mathcal{C})^{\flat}$ are injective, while the mappings
	$\leftarrow$ from $\mathcal{C}^{\flat}$ and the mappings $\rightarrow$ from
	$(C_{d}/\mathcal{C})$ are surjective. Thus, absolute and relative cohomology
	are structurally similar but qualitatively different from absolute homology
	and relative homology.
\end{example}

\begin{figure}
\centering
\begin{tikzcd}
                                  & \vdots \arrow[d, "\partial_{n+2}"]                                   & \vdots \arrow[d, "\partial_{n+2}"]                               & \vdots \arrow[d, "\partial_{n+2}"]                                            & \vdots \arrow[d, "\partial_{n+2}"]                                   &        \\
\cdots \arrow[r, "f^{i-2}_{n+1}"] & C_{n+1}^{i-1} \arrow[r, "f^{i-1}_{n+1}"] \arrow[d, "\partial_{n+1}"] & C_{n+1}^{i} \arrow[r, "f^{i}_{n+1}"] \arrow[d, "\partial_{n+1}"] & C_{n+1}^{i+1} \arrow[r, "f^{i+1}_{n+1}", no head] \arrow[d, "\partial_{n+1}"] & C_{n+1}^{i+2} \arrow[d, "\partial_{n+1}"] \arrow[r, "f^{i+2}_{n+1}"] & \cdots \\
\cdots \arrow[r, "f^{i-2}_{n}"]   & C_{n}^{i-1} \arrow[d, "\partial_{n}"] \arrow[r, "f^{i-1}_{n}"]       & C_{n}^{i} \arrow[d, "\partial_{n}"] \arrow[r, "f^{i}_{n}"]       & C_{n}^{i+1} \arrow[d, "\partial_{n}"] \arrow[r, "f^{i+1}_{n}"]                & C_{n}^{i+2} \arrow[d, "\partial_{n}"] \arrow[r, "f^{i+2}_{n}"]       & \cdots \\
\cdots \arrow[r, "f^{i-2}_{n-1}"] & C_{n-1}^{i-1} \arrow[d, "\partial_{n-1}"] \arrow[r, "f^{i-1}_{n-1}"] & C_{n-1}^{i} \arrow[d, "\partial_{n-1}"] \arrow[r, "f^{i}_{n-1}"] & C_{n-1}^{i+1} \arrow[d, "\partial_{n-1}"] \arrow[r, "f^{i+1}_{n-1}"]          & C_{n-1}^{i+2} \arrow[d, "\partial_{n-1}"] \arrow[r, "f^{i+2}_{n-1}"] & \cdots \\
                                  & \vdots                                                               & \vdots                                                           & \vdots                                                                        & \vdots                                                               &       
\end{tikzcd}
\caption[A persistence complex]{A persistence complex, where moving to the right increases the filtration index, while moving downwards decreases the dimension.}
\end{figure}

\begin{theorem}
	{(Persistence Partition) \cite[\S 2.6]{de2011dualities}} \label{persistence
	partition} Let $\mathcal{C}$ be given and $\partial$ as above, then there
	exists a partition
	\begin{align*}
		\{1,2, \ldots, d\} = P^{\Join} \sqcup P \sqcup N
	\end{align*}
	with a bijective pairing $P \leftrightarrow N$, such that
	\begin{align}
		p \text{ is paired with }n \; \iff [p,n) \in \textbf{Pairs}(\mathcal{C}, \partial).
	\end{align}
	Furthermore, there is a basis
	$\hat{\sigma}_{1}, \hat{\sigma}_{2}, \ldots, \hat{\sigma}_{d}$ of $C_{d}$ such
	that
	\begin{enumerate}
		\item $C_{i} = \langle \hat{\sigma}_{1}, \ldots, \hat{\sigma}_{i} \rangle$ for
			each $i$.

		\item $\partial\hat{\sigma}_{p} = 0$ for all $p \in P^{\Join}$.

		\item $\partial\hat{\sigma}_{n} = \hat{\sigma}_{p}$, and thus
			$\partial\hat{\sigma}_{p} = 0$, for all $[p,n) \in \textbf{Pairs}$.
	\end{enumerate}
	It follows that the persistence diagram
	$\textbf{Pers}(H(\mathcal{C},\partial))$ consists of $[a_{p}, \infty)$ for
	$p \in P^{\Join}$ together with intervals $[a_{p},a_{n})$ for
	$[p,n) \in \textbf{Pairs}$.
\end{theorem}

\begin{proof}
	Let $\mathcal{C}= \{C_{i}\}_{i \in I}$ be a filtered chain complex associated with
	a topological space, structured over a field $\mathbb{F}$ for any finite index
	set $I$. The filtration indices ${1, 2, \ldots, d}$, where $d = \vert I \vert$,
	classify the stages at which elements are born or die in homology. The
	partition of $\{1, 2, \ldots, d\}$ into $P^{\Join} \sqcup P \sqcup N$ is then
	constructed such that:
	\begin{itemize}
		\item $P^{\Join}$ contains indices corresponding to elements that persist indefinitely,
			i.e., those for which $\partial\sigma_{i} = 0$ and $\sigma_{i}$ does not
			become a boundary at any higher index.

		\item $P$ and $N$ are paired bijectively, where each $p \in P$ (births)
			corresponds uniquely to an $n \in N$ (deaths), signifying the termination of
			the homological feature introduced at $p$. This pairing is determined
			through the matrix reduction of $\partial$, ensuring that every cycle born
			at stage $p$ becomes a boundary at stage $n$.
	\end{itemize}

	We then construct a basis ${\hat{\sigma}_i}$ such that for each $i$, $\hat{\sigma}
	_{i}$ is a generator of $C_{i}$. Specifically, if $i \in P^{\Join}$, then
	$\partial\hat{\sigma}_{i} = 0$, indicating these elements are cycles that persist
	indefinitely. For pairs $[p, n) \in \textbf{Pairs}(\mathcal{C}, \partial)$
	with $p \in P$ and $n \in N$, set $\partial\hat{\sigma}_{n} = \hat{\sigma}_{p}$
	and $\partial\hat{\sigma}_{p} = 0$, reflecting the fact that the cycle
	$\hat{\sigma}_{p}$ born at $p$ becomes a boundary at $n$ and thus ceases to
	contribute to homology past $n$.

	Hence, the persistence diagram $\textbf{Pers}(H(\mathcal{C}, \partial))$ contains
	intervals $[a_{p}, \infty)$ for each $p \in P^{\Join}$, indicating persistent
	homological features and intervals $[a_{p}, a_{n})$ for each
	$[p, n) \in \textbf{Pairs}$, describing features with finite lifespans,
	starting as a cycle at $p$ and terminating as a boundary at $n$.
\end{proof}

In \ref{persistence partition} the index sets are defined as follows \cite[p.8]{de2011dualities}:
\begin{itemize}
	\item $P$ identifies the positive simplices that remain unpaired,

	\item $P^{\Join}$ identifies the positive simplices that do become paired,

	\item $N$ identifies the negative simplices.
\end{itemize}

The vectors $\hat{\sigma}_{p}$ and $\hat{\sigma}_{n}$ are cycles characterized by
their leading terms $\sigma_{p}$ and $\sigma_{n}$ respectively, while the vector
$\hat{\sigma}_{n}$ is a chain with leading term $\sigma_{n}$. This chain 'kills'
the homology class of its paired cycle $\hat{\sigma}_{p}$ through the boundary relation
$\partial \hat{\sigma}_{n} = \hat{\sigma}_{p}$.

\begin{theorem}
	For all integers $d \geq 0$ it holds that \cite[\S 2.4]{de2011dualities}:
	\begin{align*}
		\textbf{Pers}(H_{d}(\mathcal{X}) \cong \textbf{Pers}(H_{d+1}(\mathcal{X}),                       \\
		\textbf{Pers}(H_{d}(X_{\infty}, \mathcal{X}) \cong \textbf{Pers}(H_{d}(X_{\infty}, \mathcal{X}).
	\end{align*}
\end{theorem}

\begin{proof}
	Die Zerlegung
\end{proof}

\begin{remark}
	In this case, we get an isomorphism of multisets. This is due to the identification
	of the intervals $[a,\infty) \leftrightarrow [-\infty, a)$ for $\textbf{Pers}_{\infty}$.
	Thus, persistent homology and relative homology barcodes carry the same information,
	with a dimension shift for the finite intervals \cite[\S 2.4]{de2011dualities}.
\end{remark}

\subsection{Cohomology of Chain Complexes}

\section{Distances and the Stability Theorem}


